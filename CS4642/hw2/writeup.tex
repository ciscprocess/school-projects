%%%%%%%%%%%%%%%%%%%%%%%%%%%%%%%%%%%%%%%%%
% Short Sectioned Assignment
% LaTeX Template
% Version 1.0 (5/5/12)
%
% This template has been downloaded from:
% http://www.LaTeXTemplates.com
%
% Original author:
% Frits Wenneker (http://www.howtotex.com)
%
% License:
% CC BY-NC-SA 3.0 (http://creativecommons.org/licenses/by-nc-sa/3.0/)
%
%%%%%%%%%%%%%%%%%%%%%%%%%%%%%%%%%%%%%%%%%

%----------------------------------------------------------------------------------------
%	PACKAGES AND OTHER DOCUMENT CONFIGURATIONS
%----------------------------------------------------------------------------------------

\documentclass[paper=a4, fontsize=11pt]{scrartcl} % A4 paper and 11pt font size

\usepackage[T1]{fontenc} % Use 8-bit encoding that has 256 glyphs
\usepackage{fourier} % Use the Adobe Utopia font for the document - comment this line to return to the LaTeX default
\usepackage[english]{babel} % English language/hyphenation
\usepackage{amsmath,amsfonts,amsthm} % Math packages

\usepackage{lipsum} % Used for inserting dummy 'Lorem ipsum' text into the template

\usepackage{sectsty} % Allows customizing section commands
\allsectionsfont{\centering \normalfont\scshape} % Make all sections centered, the default font and small caps

\usepackage{graphicx}
\usepackage{fancyhdr} % Custom headers and footers
\pagestyle{fancyplain} % Makes all pages in the document conform to the custom headers and footers
\fancyhead{} % No page header - if you want one, create it in the same way as the footers below
\fancyfoot[L]{} % Empty left footer
\fancyfoot[C]{} % Empty center footer
\fancyfoot[R]{\thepage} % Page numbering for right footer
\setlength{\headheight}{13.6pt} % Customize the height of the header

\numberwithin{equation}{section} % Number equations within sections (i.e. 1.1, 1.2, 2.1, 2.2 instead of 1, 2, 3, 4)
\numberwithin{figure}{section} % Number figures within sections (i.e. 1.1, 1.2, 2.1, 2.2 instead of 1, 2, 3, 4)
\numberwithin{table}{section} % Number tables within sections (i.e. 1.1, 1.2, 2.1, 2.2 instead of 1, 2, 3, 4)

\setlength\parindent{0pt} % Removes all indentation from paragraphs - comment this line for an assignment with lots of text

%----------------------------------------------------------------------------------------
%	TITLE SECTION
%----------------------------------------------------------------------------------------

\newcommand{\horrule}[1]{\rule{\linewidth}{#1}} % Create horizontal rule command with 1 argument of height

\title {	
\normalfont \normalsize 
\textsc{Georgia Institute of Technology, College of Computing} \\ [25pt] % Your university, school and/or department name(s)
\horrule{0.5pt} \\[0.4cm] % Thin top horizontal rule
\huge CX4640: Homework 2 \\ % The assignment title
\horrule{2pt} \\[0.5cm] % Thick bottom horizontal rule
}

\author{Nathan Korzekwa} % Your name

\date{\normalsize\today} % Today's date or a custom date

\begin{document}

\maketitle % Print the title

%----------------------------------------------------------------------------------------
%	PROBLEM 1
%----------------------------------------------------------------------------------------

\section*{Problem 2.7}
``I don't wanna work, I don't wanna work; I just want to bang on this mug all day!'' -- Michael Scott \\


\subsection*{Part A}
\begin{align}
det(A) &= 1 - (1 + \epsilon)(1 - \epsilon) \\
	   &= \epsilon^2	
\end{align}

\subsection*{Part B}
So for the determinant to be greater than 0, 
\begin{align}
	0 \leq \epsilon < \epsilon_{mach}
\end{align}

\subsection*{Part C}
$$A =
\left[\begin{array}{cc}
1 & 1 + \epsilon\\
1 - \epsilon & 1\\
\end{array}\right]$$ \\*

$$M_1A =
\left[\begin{array}{cc}
1 & 1 + \epsilon\\
0 & \epsilon^2\\
\end{array}\right] = U$$ \\*

$$M_1 =
\left[\begin{array}{cc}
1 & 0\\
-1 + \epsilon & 1\\
\end{array}\right]$$ \\*

$$M_1^{-1} =
\left[\begin{array}{cc}
1 & 0\\
1 - \epsilon & 1\\
\end{array}\right] = L$$ \\*

\subsection{Part D}
\begin{align*}
	det(U) &= \epsilon^2 - 0 \\
		   &= \epsilon^2 \\
\end{align*}
\begin{align*}
	0 \leq \epsilon < \sqrt{\epsilon_{mach}}
\end{align*}

%----------------------------------------------------------------------------------------
%	PROBLEM 2
%----------------------------------------------------------------------------------------

\section*{Problem 2.21}
\begin{align*}
	\vec{x} = B^{-1}(2A + I)(C^{-1} + A)\vec{b} \\
	B\vec{x} = 2AC^{-1}\vec{b} + 2A^2\vec{b} + C^{-1}\vec{b} + A\vec{b}
\end{align*}

From this conclusion, we can solve two linear systems since $x^\prime = C^{-1}\vec{b}$ can be solved for with $Cx^\prime = \vec{b}$, without computing any inverses. The implementation for this question can be found in the included file ``q2.m''.

%------------------------------------------------

%----------------------------------------------------------------------------------------
%	PROBLEM 3
%----------------------------------------------------------------------------------------

\section*{Problem 2.26}
\subsection*{Part A}
Because the columns of $uv^T$ are multiples of each other, the only thing we need to worry about is annihilating one of the diagonals, and therefore degrees of freedom of the matrix.

Ergo, for an $n \times n$ matrix
$$
	u_iv_i \neq 1; 0 \leq i < n - 1
$$

\subsection*{Part B}
By the Sherman-Morrison formula, 
$$(A - \vec{u}\vec{v}^T)^{-1} = A^{-1} + A^{-1}\vec{u}(1  - v^TA^{-1}u)^{-1}v^TA^{-1}$$
$$(I - \vec{u}\vec{v}^T)^{-1} = I^{-1} + I^{-1}\vec{u}(1  - v^TI^{-1}u)^{-1}v^TI^{-1}$$
$$(I - \vec{u}\vec{v}^T)^{-1} = I + \vec{u}(1  - v^Tu)^{-1}v^T$$
And since $1 - v^Tu$ is a scalar,
$$(I - \vec{u}\vec{v}^T)^{-1} = I + \sigma\vec{u}v^T$$
where $\sigma = (1  - v^Tu)^{-1}$.

So if we want to write $(I - \vec{u}\vec{v}^T)^{-1} = I - \sigma\vec{u}v^T$, we factor the negative into $\sigma$:

$$
\sigma = (v^Tu - 1)^{-1}
$$.

\subsection*{Part C}
Yes, an elementary elimination matrix is, in fact, elementary because it differs from the identity matrix by only one column. Because of this, the ``perturbation matrix'' (the difference between the identity and the elimination matrix) is rank one since all of its columns are multiples of each other (linearly dependent).

Because of this, the elementary elimination matrix can be represented as:
$$
	M_k = I - \vec{m}\vec{e}_k^T 
$$
where 
$$
m = (0, ..., m_{k+1}, ..., m_n)^T
$$

$M_k$ eliminates all numbers below the $k$th row in a vector.
%------------------------------------------------

%----------------------------------------------------------------------------------------
%	PROBLEM 4
%----------------------------------------------------------------------------------------

\section*{Problem 4}
Using the 1-norm, both results produced low relative errors on estimating the condition number -- with the random method working better on the first matrix $A_1$ and the iterative, deterministic method from part B working better on the second test matrix, $A_2$.

This is evidenced in the program output:

>> q4 \\
A1 rel error method 1: 0.486764 \\
A1 rel error method 2: 0.644651 \\
A2 rel error method 1: 0.201978 \\
A2 rel error method 2: 0.482083 \\
>> q4 \\
A1 rel error method 1: 0.486764 \\
A1 rel error method 2: 0.066616 \\
A2 rel error method 1: 0.201978 \\
A2 rel error method 2: 0.451165 \\ 
>> q4
A1 rel error method 1: 0.486764 \\
A1 rel error method 2: 0.410768 \\ 
A2 rel error method 1: 0.201978 \\
A2 rel error method 2: 0.296503 \\

The code that runs these tests can be found in ``q4.m''.

%------------------------------------------------

\end{document}